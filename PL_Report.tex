\documentclass[autodetect-engine,dvi=dvipdfmx,ja=standard,
               a4j,11pt]{bxjsarticle}
\RequirePackage{geometry}
\geometry{reset,paperwidth=210truemm,paperheight=297truemm}
\geometry{hmargin=25truemm,top=20truemm,bottom=25truemm,footskip=10truemm,headheight=0mm}
\usepackage{graphicx}
\usepackage{fancyvrb}
\usepackage{amsmath}
\renewcommand{\theFancyVerbLine}{\texttt{\footnotesize{\arabic{FancyVerbLine}:}}}
\usepackage{newfloat}
\DeclareFloatingEnvironment[name=Listing, fileext=lol]{eopcode}
%%======== レポートタイトル等 ======================================%%
% ToDo: 提出要領に従って,適切なタイトル・サブタイトルを設定する
\title{プログラミング言語レポート}

% ToDo: 自分自身の氏名と学生番号に書き換える
\author{氏名: 赤松 佑哉 (akamatsu, YUYA) \\
        学生番号: 09B23595}

% ToDo: レポート課題等の指示に従って適切に書き換える
\date{出題日: 2024年5月9日 \\
      提出日: 2024年5月日 \\
      締切日: 2024年6月4日 \\}  % 注:最後の\\は不要に見えるが必要.


%%======== 本文 ====================================================%%
\begin{document}
\maketitle
\section{概要}
講義を通して学んだ関数型言語\verb|SML|言語を実際の問題解決を通して実践した.
今回,\verb|C|言語の標準ライブラリに存在する文字列操作関数,\verb|strcat|
,\verb|strcmp|,\verb|strcpy|,\verb|strexi|,\verb|strlen|
,\verb|sort|と同等の操作をリストに行う\verb|SML|プログラムを作成した.
いかに各関数の概要を示す.
\begin{itemize}
    \item \textbf{listcat} :\verb|strcat|に相当し,2つのリストを連結する.
    \item \textbf{listcmp} :\verb|strcmp|に相当し,2つのリストが要素・順序ともに等しいか判定する.
    \item \textbf{listcpy} :\verb|strcpy|に相当し,リストの複製(コピー)を行う.
    \item \textbf{listexi} :\verb|strexi|のように,条件を満たす要素がリスト中に存在するかを確認する(高階関数を用いる).
    \item \textbf{listlen} :\verb|strlen|に相当し,リストの長さ(要素数)を返す.
    \item \textbf{listsort}:\verb|sort|に対応し,マージソートによりリストを昇順に並べ替える.
\end{itemize}
作成したプログラムは第\ref{code}章に添付している.
\section{\texttt{listcat.sml}}
引数として二つのリストを受け取り,そのリストを繋げてできる一つのリストを返り値
とする\verb|listcat|関数を作成した.パターンマッチングを使用し引数の,場合分けを行い
異なる処理を行う.片方が空リストであればもう片方のリストを返す.どちらも要素を持つときは
第一引数の初めの要素を再帰した\verb|listcat|の戻り値に加えるように再帰呼び出ししている.
以下の入力を引数として渡し動作確認を行った.
\begin{Verbatim}
val test1 = listcat([1,2,3], [4,5])
val test2 = listcat([1,2,3], [])
val test3 = listcat([], [1,2,3])
val test4 = listcat([], [])    
\end{Verbatim}
実行結果は次のようになった.
\begin{Verbatim}
    
\end{Verbatim}

\section{\texttt{listcmp.sml}}
\section{\texttt{listexi.sml}}
\section{\texttt{listlen.sml}}
\section{\texttt{listsort.sml}}
\subsection{\texttt{listsplit.sml}}
\subsection{マージソートの実装}
\section{\texttt{SML}や講義に関する所感}

%%=================================================================%%
\section{作成したプログラムのソースコード} \label{code}
\subsection{\texttt{listcat.sml}のソースコード}
\begin{Verbatim}[numbers=left, xleftmargin=8mm, numbersep=6pt,
    fontsize=\small, baselinestretch=0.8]
fun listcat([], x) = x : int list
   |listcat(x, []) = x : int list
   |listcat(x::xs, y) = x :: listcat(xs, y)
\end{Verbatim}

\end{document}
%%=================================================================%%

