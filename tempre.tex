\documentclass[autodetect-engine,dvi=dvipdfmx,ja=standard,
               a4j,11pt]{bxjsarticle}
\RequirePackage{geometry}
\geometry{reset,paperwidth=210truemm,paperheight=297truemm}
\geometry{hmargin=25truemm,top=20truemm,bottom=25truemm,footskip=10truemm,headheight=0mm}
\usepackage{graphicx}
\usepackage{fancyvrb}
\usepackage{amsmath}
\renewcommand{\theFancyVerbLine}{\texttt{\footnotesize{\arabic{FancyVerbLine}:}}}
\usepackage{newfloat}
\DeclareFloatingEnvironment[name=Listing, fileext=lol]{eopcode}
%%======== レポートタイトル等 ======================================%%
% ToDo: 提出要領に従って,適切なタイトル・サブタイトルを設定する
\title{}

% ToDo: 自分自身の氏名と学生番号に書き換える
\author{氏名: 赤松 佑哉 (akamatsu, YUYA) \\
        学生番号: 09B23595}

% ToDo: レポート課題等の指示に従って適切に書き換える
\date{出題日: 2024年月日 \\
      提出日: 2024年月日 \\
      締切日: 2024年月日 \\}  % 注:最後の\\は不要に見えるが必要.


%%======== 本文 ====================================================%%
\begin{document}
\maketitle
\section{}
\begin{Verbatim}
\end{Verbatim}
\begin{table}[b]
    \centering 
      \caption{計算結果}%表の題%
      \begin{tabular}{|l|l|l|}
      \hline
        \textbf{} & \textbf{} & \textbf{} \\ %label
      \hline
        \verb||    &   $1$  &   $1$  & \\
      \hline
      \end{tabular}
    \end{table}
    \begin{figure}[htb]
        \centering
        \rotatebox{270}{\includegraphics[scale=0.4]{Pro_Lan_Result.PNG}}
        \vspace{20pt} % ← 適宜調整
        \caption{}
    \end{figure}
\section{作成したプログラムのソースコード}
\begin{Verbatim}[numbers=left, xleftmargin=8mm, numbersep=6pt,
    fontsize=\small, baselinestretch=0.8]
\end{Verbatim}

\end{document}