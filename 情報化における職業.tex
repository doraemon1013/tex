\documentclass[autodetect-engine,dvi=dvipdfmx,ja=standard,
               a4j,11pt]{bxjsarticle}
\RequirePackage{geometry}
\geometry{reset,paperwidth=210truemm,paperheight=297truemm}
\geometry{hmargin=25truemm,top=20truemm,bottom=25truemm,footskip=10truemm,headheight=0mm}
\usepackage{graphicx}
\usepackage{fancyvrb}
\usepackage{amsmath}
\usepackage{tabularx}
\renewcommand{\theFancyVerbLine}{\texttt{\footnotesize{\arabic{FancyVerbLine}:}}}
\usepackage{newfloat}
\DeclareFloatingEnvironment[name=Listing, fileext=lol]{eopcode}
\topmargin=-1cm
    \textheight=24cm
    \textwidth=15.5cm
    \oddsidemargin=-.2cm
    \evensidemargin=-.2cm
%%======== レポートタイトル等 ======================================%%
% ToDo: 提出要領に従って,適切なタイトル・サブタイトルを設定する
\title{{\normalsize 情報化における職業 レポート(第二回)
        \\[1.5ex]
       {\LARGE 講師: 中島耕太先生} \\[1.5ex]}}

% ToDo: 自分自身の氏名と学生番号に書き換える
\author{氏名: 赤松 佑哉 (akamatsu, YUYA) \\
        学生番号: 09B23595\\
        E-mail: p5785hcf@s.okayama-u.ac.jp}

% ToDo: レポート課題等の指示に従って適切に書き換える
      \date{提出日: 2025年7月11日 \\
      締切日: 2025年8月3日 \\}  % 注:最後の\\は不要に見えるが必要.


%%======== 本文 ====================================================%%
\begin{document}
\maketitle
\section{授業まとめ}
この授業では富士通株式会社から中島耕太先生にお越しいただき,これの会社での研究であったり経験であったりをお話しいただいた.
中島先生は富士通株式会社での研究を通して,情報化社会において情報工学がどのように社会に役立っているか,情報技術者がどのように働いているか,
将来向けてどのようなことをかんがえればよいかについて重点的にお話ししてくださった.導入として
中島先生の研究ないようについてお話しいただき,とくに富岳を中心とした事例を紹介してもらった\\
AIは大量のデータを高速かつ効率的に処理する能力が不可欠であり,従来のコンピューターでは限界がある中,スーパーコンピューターでは その計算能力と並列処理性能により
AIの開発の応用として活用されていることを学んだ.先生の実践例としては,深層学習モデルの訓練や生成AIの開発,大規模
言語もモデルの構築などAIの研究開発においてスーパーコンピューターがどのように活用されているかを紹介していただいた.そして,この分野に関する研究は,科学研究,医療,気象予測
,創薬,社会シミュレーションなど幅広い分野に応用されており,社会に大きな影響を与えていることを学んだ.
そのスーパーコンピュータの中でも富岳は少し特殊であり,その特徴としては,GPUに依存せず国産CPUを使用していることが挙げられた.
現在富岳を含めた世界のスーパーコンピュータランキングであるTOP500では,富岳は2020年6月から2022年11月までの間,世界一位を獲得していた.
現在はアメリカのスーパーコンピュータが上位を独占しているが,どれもGPU依存のスーパーコンピュータでありCPUの使用している富岳はほかにはないメリットデメリットが存在する.
その富岳をもちいたAI応用事例としては,がんをはじめとする疾患における遺伝子間の因果関係の解明は、従来の計算機環境では極めて困難であり、
例えば約2万個のヒト遺伝子についての全組み合わせ解析には数千年の時間が必要とされていた。
「富岳」はその並列処理性能を活かし、膨大な遺伝子ネットワークの因果探索をわずか1日で実現。
これにより、胃がんをはじめとする疾患との新たな関連遺伝子の発見が進み、創薬や予防医療の分野にも大きな影響を与えている.ほかにも
日本語は語順が柔軟で主語が省略されることも多く、英語に比べて自然言語処理の難易度が高いと言われている。「富岳」はその高い処理能力により、
日本語のニュース記事・行政文書・会話文といった多様な文体を対象とした大規模言語モデル(LLM)の学習を支援,
といった事例も紹介していただいた.一番印象に残っているのは触媒解析による環境エネルギー分野への貢献である.
脱炭素社会の実現に向けた次世代燃料として注目される「アンモニア」の生成過程においては、有効な触媒の発見が鍵を握る。「富岳」を活用することで、量子化学的なシミュレーションを大量に並列実行し、
多数の候補物質を高速解析することが可能となった。これにより、実験による試行錯誤の回数を劇的に削減し、触媒材料の探索期間も大幅に短縮されている。
このように,一般の高性能PCを利用しても到底太刀打ちできないような問題もスーパーコンピュータを利用することで解決できることを学んだ.\\
スーパーコンピュータに限らず情報国学が支える社会インフラはたくさんある.例えば,検索エンジン,SNS,スマートフォンといった日常的サービス,銀行
,証券取引,行政システムといった社会基盤系といったものが挙げられる.
つづいて,社会インフラの分類だけでなく情報技術者の分類と役割についてもお話しいただいた.
講義では,研究,開発,SEの三種類の代表的な職種についてその代表的な職種についてその役割の違いが明確に示された.
研究分野では,新技術の開発など先端技術の探求が主である.開発では,製品の実装,共通部品課など安定した技術の応用と実装が求められる.
反対にSEでは,顧客向けのシステムの設計の構築,運用保守など顧客のニーズに応えることが求められる.
大まかな分類すぎるところもあるが,将来の職種を選ぶ際にこれらの職種の特徴からどのようなタイプの職業が自分に合っているのかを
見極め,選定することが大事であることを理解した.大学院の研究と異なり,企業の研究開発では,研究の成果を社会に還元することが求められる.
術者としての将来を考えるにあたり、自らがどのような専門性を身につけ、どのような企業でどんな働き方を望むかを見極めることは極めて重要であるので
学部・修士・博士の違いと技術習得の段階についてもお話しいただいた.情報という分野はほかの分野と違って,大学院への進学のメリットがとても大きい.
技術だけでなく、伝達力・協調性・倫理観など、社会と関わる上での人間的な力も必要不可欠だと感じた.
\\
学力,市場価値の他には心構えについてもお話しいただいた.その中で特に印象に残っているのは
「ともかくやってみる」という姿勢を持ち、挑戦を継続すること,社会と技術の関係を常に意識し、実装可能なかたちで知識を社会に還元すること,
これらの精神は時代が進んでいっても変わらない技術者として重要な心構えだと再確認した.
\section{感想}
今回の講義を通じて、情報工学という分野が社会のあらゆるところで深く関係していること、そしてその技術を支える情報技術者の仕事が非常に多様で奥深いものであることを実感した。
情報技術者としてのキャリア形成においては、企業選びや専門性の確立が重要であることも学んだ。
中島先生も入社して,私たちが想像する研究社のような仕事を職をするをするまで4--6年かかったとおっしゃっていた.
論文執筆ができるようになり,特許も発行できるようになり,会社の仕組みを理解し,社会に還元できるようになるまでには時間がかかることを実感した.
逆に言うと,初めの企業選びは非常に重要であり、企業の研究開発の方針や文化が自分の目指す方向と合致しているかを見極めることが大切である。それができると,あとは
必要な技術を身につけていくことに集中できる.難しいが求められることはシンプルであると感じた.
そのため、企業選びは慎重に行うべきであり、特に研究開発職を目指す場合は、企業の研究開発の方針

中島先生のこれまでのキャリアは、一人の技術者が専門性と挑戦心によって社会に大きな影響を与えることを証明していると感じた。また、「ともかくやってみる」「研究は楽しい」というメッセージには非常に勇気づけられた。情報技術に関する仕事は高度で大変な一方、社会課題の解決や人々の暮らしの向上につながることから、大きなやりがいと責任が伴っている。それだけに、努力と挑戦を続ければ、周囲の協力も得ながら成果を生み出せるという実体験を聞けたことは、これから技術者として就職しようとする自分にとって心強かった。
スーパーコンピューターによるAI技術への貢献では、遺伝子解析、自然言語処理、環境分野への応用、生成AI基盤といった例が挙げられ、情報工学が今後さらに重要な役割を果たすことがよく分かった。これらはすべて高度なコンピューティング技術に支えられており、演算性能や通信性能の限界をいかに突破するかが社会の技術進歩のスピードに直結していることを実感した。
最後に、「キャリア形成では会社を見比べるべき」「きちんと伝える力が重要」といった指摘は、技術だけでなく人間としての成長にもつながると感じた。今回の学びを活かし、今後はより深く情報工学を探求しながら、自分自身も社会に貢献できる技術者を目指して努力していきたい。
\end{document}